

In this appendix,  we show that \begin{equation}  \lim_{r \rightarrow 0}  R_p (r) =  2p -1, \end{equation}
for an irrotational and isotropic velocity field, 
where  $ R_p (r ) $ is the ratio between the longitudinal and transverse structure functions of order $ p $, defined in (\ref{Ratio}).
We start by considering $ p = 2 $.   If $ u $ and $ v $ are the velocity component in the $ x $- and $ y $-directions, respectively, it is clear that
\begin{equation}
 \lim_{r \rightarrow 0}  R_2 (r) = \frac{\left \langle \frac{\partial u}{\partial x}  \frac{\partial u}{\partial x}  \right \rangle} {\left \langle \frac{\partial v}{\partial x}  \frac{\partial v}{\partial x}  \right \rangle} .
\end{equation} 
Since the velocity field is irrotational we can write it as $ {\bf u} = \bnabla \phi $, where $ \phi $ is the velocity potential. By statistical homogeneity we find
\begin{equation} \label{Tensor}
\left \langle \frac{\partial u_i}{\partial x_j}  \frac{\partial u_k}{\partial x_l}  \right \rangle = \left \langle  \frac{\partial ^2 \phi}{\partial x_i \partial x_j}  \frac{\partial ^2 \phi}{\partial x_k\partial x_l} \right  \rangle = 
\left \langle \phi \frac{\partial^4 \phi}{\partial x_i \partial x_j \partial x_k \partial x_l }\right \rangle 
\end{equation} 
The fourth order tensor on the left hand side of (\ref{Tensor}) is 

