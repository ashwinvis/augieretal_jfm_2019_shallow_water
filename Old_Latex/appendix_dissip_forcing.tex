In this appendix, we give the expressions of the kinetic energy
dissipation and forcing rates, which are unusual for the one-layer
shallow-water model since the kinetic energy has not a quadratic
expression.

Since only the ageostrophic variable is forced ($\hat f_d = 0$),%
(i) only the thickness fluctuation $\eta$ is forced in the
non-rotating case and %
(ii) the force terms in (\ref{eqapmnum}) are simply equal to $\hat
f_\pm = \hat f_a/2$.
%
In order to compute the force $\hat f_a$, we start from a
pre-normalized force $\hat f_{0}$ obtained from a normal random
process decorrelated in time.
%
The force is then normalized such that the quadratic part of the total
energy $\langle|\uu|^2 + c^2 \eta^2\rangle/2$ is injected at a
constant rate.
%
The spectral injection rate of the quadratic energy averaged over one
time step is (see appendix~\ref{app_comp})
\begin{equation}
P_q(\kk, t) 
= 
\Re\{ \hat \uu(\kk)^* \cdot \hat \ff(\kk) 
+
c^2 \hat \eta(\kk)^* \hat f_\eta(\kk) \}
+ 
(|\hat\ff|^2 + c^2 |\hat f_\eta|^2) \delta t /2,
\end{equation}
where $\Re$ denotes the real part.
%
Writing $\hat f_a = \alpha \hat f_{0}$, it is straightforward to solve
a second-order equation for the coefficient $\alpha$ in order to fix
the injection rate $\Add{P_0 \equiv} \sum_\kk P_q(\kk, t)$ to unity.



We consider viscous operators such as $\p_t \uu |_{\mbox{\tiny diss}}
=- \nu_n i^{-n} \bnabla^n \uu $ and $\p_t \eta |_{\mbox{\tiny diss}}
=- \nu_n i^{-n} \bnabla^n \eta $, with n even.  The values $n=-4$, 0,
2 and 8 correspond to hypo-viscosity, linear damping, newtonian
viscosity and hyper-viscosity, respectively.  In this study, we have
only used the value $n=8$.
%
Note that the dissipation operator is applied on $\eta$ and not on
$h$.  Using $\p_t \JJ |_{\mbox{\tiny diss}} = h \p_t \uu
|_{\mbox{\tiny diss}} + \uu \p_t h |_{\mbox{\tiny diss}} = - \nu_n
i^{-n} (h \bnabla^n \uu + \uu \bnabla^n \eta )$, the KE dissipation
rate can be computed as
\begin{eqnarray}
\p_t \meanx{E_K} |_{\mbox{\tiny diss}} 
&=& 
\mean{
\p_t \uu |_{\mbox{\tiny diss}} \cdot \JJ
+ \uu \cdot \p_t \JJ |_{\mbox{\tiny diss}}
} /2, \\
&=& 
- \nu_n i^{-n}
\mean{
\JJ \cdot \bnabla^n \uu + |\uu|^2 \bnabla^n \eta /2
} \\
% &=& 
% - \sum_\kk \hat f_{dn} \left[ \scalarprod{\JJ}{\uu}
% + \scalarprod{|\uu|^2/2}{\eta}
% \right]\\
&=& 
- \sum_\kk 2 f_{dn}  E_K(\kk)
- \sum_\kk  f_{dn} \scalarprod{|\uu|^2/2}{\eta},
\end{eqnarray}
where $ f_{dn} (\kk) = \nu_n |\kk|^n$ is the dissipative frequency.
The first term of the rhs is the usual term but there is also an
additional term that is not negatively defined.




The spectral injection rate of quadratic kinetic energy averaged over
one time step is
\begin{equation}
P_{K}(\kk, t) 
= \frac{1}{\delta t} \int_t^{t+\delta t} dt'
\scalarprod{\uu(t')}{\ff}
= 
\scalarprod{\uu}{\ff}
+ 
|\ff|^2 \delta t /2,
\end{equation}
where we have used the fact that the forcing $\ff$ is constant over
the time step.
%
In order to calculate the total KE injection rate, we have to take
into account the non-quadratic term in the expression of the kinetic
energy.
%
The forcing terms are $\p_t \uu |_f = \ff$, $\p_t h |_f = f_h$ and $
\p_t \JJ |_f = \ff_\JJ = h \ff + \uu f_h$ so that the instantaneous
injection rate can be written as
\begin{equation}
 P_{K\mbox{\tiny inst}}(\kk, t) 
= \p_t  E_K(\kk, t)|_f 
= \scalarprod{\JJ}{\ff}/2 + \scalarprod{\uu}{\ff_\JJ}/2.
\end{equation}
Averaging over one time step, we obtain
\begin{eqnarray}
 P_{K}(\kk, t) 
&=& \frac{1}{\delta t} \int_t^q{t+\delta t} 
dt'  P_{K\mbox{\tiny inst}}(\kk, t') 
\\
&=& \frac{1}{\delta t} \int_t^{t+\delta t} dt'
\left[
\scalarprod{\JJ(t')}{\ff}/2
+ \scalarprod{\uu(t')}{h(t') \ff + \uu(t') f_h}/2 
\right].
\end{eqnarray}
Using the estimates $\uu(t') = \uu(t)+ \ff (t'-t)$, $h(t') = h(t)+ f_h
(t'-t)$ and $\JJ(t') = \JJ(t)+ \ff f_h (t'-t)^2$, the KE injection
rate averaged over one time step can be computed at the leading order
as
\begin{eqnarray}
 P_{K}(\kk, t) \simeq 
& & \scalarprod{\JJ}{\ff}/2 + \scalarprod{\uu}{\ff_\JJ}/2 \nonumber\\
&+& \left[  
\scalarprod{\ff}{\ff_\JJ}/2  
+ \scalarprod{\uu}{f_h\ff} 
\right] \delta t /2.
\end{eqnarray}
