

In order to analyze the spectral flux of energy, we derive the spectral energy
budget, i.e.\ the governing equations for spectral KE and APE
functions. This is not as straightforward as in incompressible turbulence since
the expression of the kinetic energy  is not
quadratic. To define the spectral KE function, we use the relation
\begin{equation}
\mean { E_K } = \sum_\kk \scalarprod{\uu}{\JJ}/2,
\end{equation}
where
\begin{equation}
\scalarprod{\baa}{\bb} \equiv \Re\{ \hat \baa(\kk)^* \cdot \hat \bb(\kk) \},
\end{equation}
and $\Re$ denotes the real part, $\kk$ is the wave number and the
hat denotes the Fourier transform.
%
The spectral KE function can therefore be defined as
\begin{equation}
E_K(\kk) \equiv \scalarprod{\uu}{\JJ}/2.
\end{equation}
Similarly, the APE can be written as the sum over all wave numbers of
the spectral APE function
\begin{equation}
E_A(\kk) =  c^2 \scalarprod{\eta}{\eta}/2 \, .
\end{equation}
The inviscid equation for the spectral KE and APE functions can be written as
\begin{eqnarray}
\p_t E_K(\kk) &=& T_K(\kk) - C (\kk),\\
\p_t E_A(\kk) &=& T_A(\kk) + C (\kk),
\end{eqnarray}
where
\begin{equation}
T_K(\kk)
= -\scalarprod{\bnabla \cdot \uu}{E_K}
-\scalarprod{\uu }{\bnabla E_K } \, ,
\end{equation}
\begin{equation}
T_A(\kk) = -c^2 \scalarprod{\bnabla \eta }{\JJ}- c^2\scalarprod{\eta}{\bnabla \cdot \JJ}  \, ,
\end{equation}
are the spectral transfer functions of KE and APE, respectively, and
\begin{equation}
C (\kk)
= \scalarprod{\uu}{\bnabla E_P}
\end{equation}
is the spectral KE-APE conversion function.
The spectral energy flux functions of KE and APE are defined as
\begin{eqnarray}
\Pi_K (k) = \sum_{| {\bf k}^{\prime} | > k} T_K ( {\bf k}^\prime ) \, , \;\;\;  \Pi_A (k) = \sum_{| {\bf k}^{\prime} | > k} T_A ( {\bf k}^\prime )
\end{eqnarray}


We will now derive an analogue of the so called `four-fifths law'
\cite[]{Kolmogorov1941} for the the third order velocity structure function of
incompressible isotropic turbulence. In the case of irrotational shallow water
wave turbulence the analogous law contains a third order structure function
involving both velocity and displacement increments. In order for such a law to
exist it is necessary that the velocity field is either non-divergent
(incompressible) or irrotational. To derive the law it is convenient to start
from the following equations
\begin{eqnarray} \label{IRR}
\p_t \uu &=& -\bnabla \cdot (\uu \cdot \uu /2) - c^2 \bnabla h  \\
\p_t \JJ &=& -\p_j (u_j \JJ) - \bnabla E_p,\\
\p_t h   &=& -\bnabla \cdot \JJ \, ,
\end{eqnarray}
where we have used the assumption that the velocity field is irrotational in
(\ref{IRR}). We consider two points, positioned at $ \bf x $ and $ {\bf
x}^\prime $, separated by the vector $ {\bf r} = {\bf x}^\prime - {\bf x} $.
Using Cartesian tensor notation we denote derivatives with respect to $ {\bf x} $
as $ \partial_i $ and derivatives with respect to $ {\bf x}^\prime $ as $
\partial^{\prime}_i $. Treating $ \bf x $ and $ {\bf x}^\prime $ as independent
variables we find

\begin{equation}
\p_t ( J_i' u_i ) = - \p_i( J_i'|\uu|^2/2) - \p_i'(u_j J_j' u_i')
-\p_i (J_i' c^2 h)  - \p_i' (u_i E_p')
\end{equation}
and
\begin{equation}
\p_t ( h' h ) = - \p_i( h'J_i ) - \p_i'( hJ_i' ),
\end{equation}
which gives
\begin{eqnarray}
\p_t ( \JJ'\cdot\uu + \JJ\cdot\uu' + 2c^2 h'h )
&=& - \p_i( J_i'|\uu|^2/2) - \p_i'( J_i|\uu'|^2/2) %\nonumber \\&&
- \p_i'(u_j J_j' u_i') - \p_i(u_j' J_j u_i) \nonumber\\
&& -c^2\p_i (J_i'h) - c^2\p_i' (J_ih') %\nonumber\\&&
- \p_i' (u_i E_p') - \p_i (u_i' E_p) \nonumber\\
&& -2c^2\p_i' (J_i'h) - 2c^2\p_i (J_ih').
\end{eqnarray}
We then assume homogeneity and take the average denoted by
$\meane{}$.   For
every function $g(\xx, \xx')$, we have $\p_i' \meane{g} =
-\p_i\meane{g} = \bnabla_{\rr} \meane{g}|_i$. We thus obtain
\begin{eqnarray}
\p_t \meane{ \JJ'\cdot\uu + \JJ\cdot\uu' + 2c^2 h'h }
&=& + \bnabla_{\rr}\cdot\meane{ \JJ'|\uu|^2/2 - \JJ|\uu'|^2/2 } \nonumber\\
&& - \bnabla_{\rr}\cdot\meane{ u_j J_j' \uu' - u_j' J_j \uu } \nonumber\\
&& - \bnabla_{\rr}\cdot\meane{ \uu E_p'- \uu' E_p } \nonumber\\
&& -c^2\bnabla_{\rr}\cdot\meane{ \JJ'h - \JJ h' }
\end{eqnarray}
%
We then introduce the structure functions and the operator $\delta$
returning the increment of a variable between two points separated by $\rr$,
for example $\delta h (\xx, \rr) = h(\xx+\rr)- h(\xx)$.
%
Using homogeneity again, we get
\begin{equation}
\meane{ (\delta h)^2 \delta\uu } =
-\meane{ h'^2 \uu } + \meane{ h^2 \uu' }
+2\meane{ hh' \uu } - 2\meane{ hh' \uu' }
\end{equation}
and
\begin{equation}
\meane{ |\delta \uu|^2 \delta\JJ } =
\meane{  |\uu|^2 \JJ' } - \meane{ |\uu'|^2\JJ }
+\meane{ u_ju_j' \JJ' } - \meane{  u_ju_j' \JJ },
\end{equation}
which gives
\begin{equation}
2\p_t \meane{ \JJ'\cdot\uu + \JJ\cdot\uu' + 2c^2 h'h }
= \bnabla_{\rr} \cdot ( \meane{ |\delta \uu|^2 \delta\JJ }
+ c^2\meane{ (\delta h)^2 \delta\uu } ).
\end{equation}
For separations that are considerably larger than dissipative scales of motion
and considerably smaller than forcing scales a standard argument \cite[see for
example][]{Frisch} gives that the left hand side is approximately equal to $ -8
\eps $, where $ \eps $ is the mean dissipation rate. It is interesting to note
that a factor of $ 8 $ appears in the case when the velocity field is
irrotational while a corresponding factor of $ 4 $ appears in the case when the
velocity field is non-divergent. Using isotropy and integrating yield the
Kolmogorov law for irrotational shallow water wave turbulence
\begin{equation}
\meane{ |\delta \uu|^2 \delta J_L }
+ c^2\meane{ (\delta h)^2 \delta u_L } = -4 \varepsilon r, \label{eq_Kolmo}
\end{equation}
where $J_L \equiv
\JJ\cdot\rr / |\rr|$ and $u_L \equiv \uu\cdot\rr / |\rr|$ are
longitudinal increments.




We will now use (\ref{eq_Kolmo}) together with straightforward geometrical and
statistical arguments to construct a model for velocity and displacement
structure functions of any order above a certain minimum. The model is similar
to the one developed by \cite{BouchaudMezardParisi1995} for Burgers equation,
although we derive it by using quite different arguments. We consider a domain
with total area $ {\mA} $ occupied by shocks which we consider as smooth lines.
We denote the total length of all shocks by $ L_s $. The mean distance between
the shocks within the domain can be estimated as
\begin{equation}
d \simeq \frac{\mA} {L_s} .
\end{equation}
A structure function of a flow variable is defined as the area average of increments  between two points separated by a distance $ r $.
Assuming that the contribution from increments where the two points are not separated by a shock is negligible for higher order structure functions we obtain
\begin{eqnarray}
\meane{|\delta h|^p}
=  \frac{1}{\mA} \int_{\mA}  d^2\xx |\delta h(\xx)|^p
 \simeq  \frac{r}{\mA} \int_\shocks ds |\sin\theta| |\Delta h (s)|^p,
\end{eqnarray}
where $\theta$ is the angle between the shock line and the separation
vector $\rr$ %
and $\Delta h (s)$ is the shock displacement amplitude.
%
Assuming isotropy, the integral can be split and we get
\begin{equation} \label{StepH}
\meane{|\delta h|^p}
\simeq
r \frac{L_s}{\mA} \meant{|\sin\theta|} \means{|\Delta h |^p}
\simeq  \frac{2}{\pi} \means{|\Delta h |^p} \frac{r}{d} ,
\end{equation}
where $\meant{}$ is the mean over $\theta$
and $\means{}$ is the mean over all shocks.

For velocity increments, we can also use a characteristic of the
velocity singularities.  The step in velocity is only in the component
perpendicular to the shocks, which implies that the longitudinal and
transverse increments can be written as  $\delta u_L = \delta u
\sin\theta$, $\delta u_T = \delta u \cos\theta$.
%
Applying the same averaging method, we obtain
\begin{equation} \label{StepL}
\meane{|\delta u_L|^p}
\simeq
\meant{|\sin\theta|^{1+p}}
\means{|\Delta u |^p} \frac{r}{d}
\end{equation}
and
\begin{equation} \label{StepT}
\meane{|\delta u_T|^p}
\simeq
\meant{|\sin\theta||\cos\theta|^{p}}
\means{|\Delta u |^p} \frac{r}{d},
\end{equation}
where $\Delta u $ is the shock velocity amplitude.
%
These assumptions completely determine the ratio between the
longitudinal and transverse structure functions, which can be computed as
\begin{eqnarray} \label{Ratio}
R_p(r) \equiv \frac{\meane{|\delta u_L|^p}}{\meane{|\delta u_T|^p}} =
\frac{\meant{|\sin\theta|^{1+p}}}{\meant{|\sin\theta||\cos\theta|^p}},
\end{eqnarray}
giving the numerical values $R_2 = 2$, $R_3 = 6\pi/8$ and $R_4 = 8/3$.
Interestingly, for an isotropic and irrotational velocity field, the longitudinal and transverse second order structure functions
are exactly related by $\meane{(\delta u_L)^2} = \p_r( r\meane{(\delta
u_T)^2} )$ \cite[]{Lindborg2007jas}. If we further assume than both second order structure functions follow
the same scaling law $\meane{ (\delta u_L)^2} \propto \meane{ (\delta
u_T)^2} \propto r^\alpha$, then we have $\meane{ (\delta u_L)^2} =
(1+\alpha) \meane{ (\delta u_T)^2} $, which gives $ R_2 = 2 $ for
$  \alpha = 1$.  In the limit $ r \rightarrow 0 $ we have $ \alpha = 2 $ and $ R_2 = 3 $.
%

In order for (\ref{StepH}-\ref{StepT}) to match with the third order structure function relation (\ref{eq_Kolmo})  the shock amplitudes should scale as
\begin{equation} \label{Strength}
 | \Delta u | \sim | c \Delta h | \sim (\varepsilon d)^{1/3} \, ,
\end{equation}
and the structure functions of order $ p $ will consequently scale as
\begin{equation} \label{StrucFunc}
\meane{|\delta u |^p}  \sim \meane{(c|\delta h |)^p} \sim  (\varepsilon  d)^{p/3} \,  \frac{r}{d} \, .
\end{equation}
Assuming that (\ref{StrucFunc}) is valid for second order moments and that the
non-quadratic contribution to the KE spectrum is negligible the corresponding
scaling law for the energy spectra reads
\begin{equation} \label{Spectra}
E_K(k)  \sim  E_A(k) \sim \varepsilon ^{2/3} d^{-1/3} k^{-2} \, .
\end{equation}
The  model  provides the following predictions for the flatness factors of velocity increments
\begin{eqnarray} \label{FL}
F_L &=&  \frac{\meane{|\delta u_L|^4}}{\meane{|\delta u_L|^2}^2}
\propto
\frac{\meant{|\sin\theta|^5}}{\meant{|\sin\theta|^3}^2} \frac{d}{r}   = \frac{9\pi}{15}  \frac{d}{r} \\ \label{FT}
F_T &=&  \frac{\meane{|\delta u_T|^4}}{\meane{|\delta u_T|^2}^2}
\propto
\frac{\meant{|\sin\theta||\cos\theta|^4}}{\meant{|\sin\theta||\cos\theta|^2}^2} =  \frac{9\pi}{10}  \frac{d}{r}
\end{eqnarray}
The ratio between the two flatness factors is equal  to
$F_T/F_L = 3/2$ according to the model.

Being such a simple model it must have limitations. First, it is obvious that
the model can only be valid for separations that fulfill the condition $ \delta
x \ll r \ll d $, where $ \delta x $ is the shock width. Second, the model
cannot be valid for low order moments or structure functions. The minimum order
for which the model may be valid is not easily determined. According to
\cite{BouchaudMezardParisi1995} structure functions of shock dominated
turbulence of all orders $ p \ge 1 $ scale linearly in $ r $. It is quite
easily argued, however, that the shocks can not determine the scaling of the
structure functions of order $ p=1 $. To see this, consider the first order
moment of the absolute value of $ \delta u_L $. The longitudinal velocity
increment is always negative over a shock, that is to say that the difference
of the projection of the velocity on the direction over which the shock is
crossed is always negative. In a domain with periodic boundary conditions the
mean value of $ \delta u_L $ is zero. The negative values of $ \delta u_L $ for
separations where a shock is crossed must therefore be compensated by positive
values for separations where no shock is crossed. There is no reason to believe
that these values are all positive, but they are only positive on average.
Thus, $ \means{ |\delta u_L | } < \langle | \delta u_L | \rangle _{ns} $, where
the left hand side is the mean value over all separations where a shock is
crossed and the right hand side is the mean value over separations where no
shock is crossed. Therefore, the model is not valid for first order moments. It
may be conjectured that it is valid for second and higher order structure
functions.
