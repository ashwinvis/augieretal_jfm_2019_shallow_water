\documentclass[12pt]{article}
\usepackage{graphicx}
 
           
\topmargin 0in % 25mm-1in makes the body start 25mm below top of paper
\headheight 3ex 
\headsep 14mm

\advance \topmargin by -\headheight
\advance \topmargin by -\headsep
     
% height of thesis textarea (11pt font):
\topskip 5ex
\textheight 225mm  
     
\oddsidemargin 	4mm     % 3cm - 1in - 1mm correction lwmech4
\evensidemargin \oddsidemargin
% width of thesis textarea (11pt font):
\textwidth 160mm
%%%%%%%%%%%%%%%%%%%%%%%%%%%%%%%%%%%%%


\parskip 0ex
\parindent 1em

\setlength{\unitlength}{1truemm}
%%%%%%%%%%%%%%%%%%%%%%%%%%%%%%%%%%%%%%%%%%%%%%%%%%%%%%%%%%%%%%%%%%%%%%%%%%%%%%
\begin{document}
\noindent {\bf \large Estimate of the characteristic shock width and lifetime in  terms of $ \nu, \; \epsilon, \; Re $ and $ Fr $.}
\vskip 5mm
\noindent Definitions
\begin{equation}
Re = \frac{\varepsilon^{1/3} L^{4/3}}{\nu} \, ,\;\;\;\;\; Fr = \frac{\varepsilon^{1/3} L^{1/3}}{c} 
\end{equation} 
The characteristic shock width, $ \delta x $, can be estimated by assuming that all dissipation takes place within the shocks and that the dissipation scale is equal to $ \delta x $. The mean dissipation over the domain area $ \cal{A} $
can thus be estimated as
\begin{equation}
\varepsilon \sim \nu  \frac{\Delta u^2} {\delta x^2} \frac{L_s \delta x} {\cal{A}}\sim \frac{\nu \varepsilon^{2/3} }{\delta x d^{1/3}} \, ,
\end{equation} 
With 
\begin{equation} \label{dScale}
d \sim Fr L   
\end{equation} 
we find that
\begin{equation}
\delta x \sim Re^{-1/4} Fr^{-1/3} \, \frac{\nu^{3/4}}{\varepsilon^{1/4}} \, ,
\end{equation}
where we recognise $ {\nu^{3/4}}/{\varepsilon^{1/4}} $ as the Kolmogorov length scale. The characteristic life time can be estimated as
\begin{equation}
\tau \sim  \frac{\Delta u^2}{\varepsilon} \frac{L_s \delta x}{A} \sim \frac{\delta x}{\; (\varepsilon d)^{1/3}} \sim Re^{-1/2} Fr^{-2/3} \left ( \frac {\nu}{\epsilon} \right )^{1/2} 
\end{equation} 
where we recognise  $ (  {\nu}/{\epsilon}  )^{1/2} $ as the Kolmogorov time scale. A necessary condition for the appearance of shocks is that $ \delta x \ll d $, leading to the condition
\begin{equation}
Re^{5/4} Fr^{4/3} \gg 1 \, .
\end{equation} 
It may be conjectured that this is not only a necessary condition but also a sufficient conditions for shocks to appear.

\end{document}