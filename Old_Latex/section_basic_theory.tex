
In a non-rotating system of reference the SW equations for a thin layer of fluid over a flat bottom and with a free
surface can be written as  \cite[see for example][]{VallisLIVRE2006}
\begin{eqnarray} \label{UEq}
\p_t \uu + \uu\cdot \bnabla  \uu
&=& - c^2 \bnabla h + \frac{\nu}{h}  \bnabla^2 \uu,   \label{eq_uu}\\ \label{HEq}
\p_t   h    &=& - \bnabla \cdot (h \uu).  \label{eq_h}
\end{eqnarray}
where $\uu$ is the horizontal velocity, $h = 1 + \eta $ the non-dimensional
thickness of the fluid layer, $ \eta $ being the surface displacement, $\nu$
the kinematic viscosity and $ c $ is the gravity wave speed. The SW equations
conserve mass, $ h $, and momentum $ {\bf J} = h{\bf u} $. As a matter of fact,
we have constructed the viscous term in (\ref{UEq}) in such a way that momentum
is conserved and energy dissipation is positive definite. This can only be
accomplished by permitting the term to be nonlinear in the flow variables.
There are different forms of nonlinear terms that fulfill both conditions and
it may be disputed which is the best one. However, in the simulations we will
use diffusion terms in both (\ref{UEq}) and (\ref{HEq}) that will not fulfill
any of these two conditions in a strict sense. Therefore, we will not discuss
this matter further. The inviscid SW equations conserve a local quantity along
trajectories of the fluid particles, the Ertel potential vorticity $\ErtelPV =
\zeta /h$, where $\zeta $ is the vorticity, which in a rotating system should
be interpreted as absolute vorticity. In this paper this conservation law will
be of no relevance since we only consider irrotational flows for which the
velocity can be written as $ \uu = \bnabla \phi $ , where $ \phi $ is the
velocity potential. In a domain with no net fluxes through the boundaries the
inviscid equations conserve total energy $ E_K + E_P $, where $ E_K = {\bf J}
\cdot {\uu} /2 $ is kinetic energy and $ E_P = c^{2} h^2/2 $ is potential
energy. The latter can be split into three terms as
\begin{equation}
E_P = c^2/2 + c^2\eta+ c^2 \eta^2/2,
\end{equation}
where the first term corresponds the potential energy of the state with no
surface displacement. The third term, which is quadratic in the surface
displacement, may be named available potential energy (APE), $E_A =
c^2\eta^2/2$, a concept introduced by \cite{Lorenz1955}. As a consequence of
mass and energy conservation the SW equations also conserve the sum of kinetic
and available potential energy, $ E_K + E_A $. The velocity potential and the
displacement of the linearized inviscid SW equations both satisfy the wave
equation,
\begin{equation}
\frac{\partial ^2 \eta}{\partial t^2} = c^2 \nabla^2 \eta \, ,
\end{equation}
with equipartition between KE and APE over a wave period in each Fourier mode.