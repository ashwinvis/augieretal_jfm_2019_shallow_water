


\subsection{Physical system and governing equations}

We consider a thin layer of fluid over a flat bottom and with a free
surface.
%
Under the assumption that the characteristic horizontal scale of the
flow $L$ is much larger than the mean layer thickness $H_0$, it can be
shown that the pressure is only a function of the local layer
thickness $H(x, y)$, i.e.\ that the flow is hydrostatic.
%
Further assuming that the horizontal flow is vertically invariant, it
can be shown that the shallow-water system is governed by the
Saint-Venant equations \cite[see for example][]{VallisLIVRE2006}
\begin{eqnarray}
(\p_t + \uu\cdot \bnabla ) \uu   
&=& - c^2 \bnabla h - f \eez \wedge \uu + \nu \bnabla^2 \uu,   \label{eq_uu}\\
\p_t   h    &=& - \bnabla \cdot (h \uu).  \label{eq_h}
\end{eqnarray}
where $\uu$ is the horizontal velocity, %
$h = H/H_0$ the normalized layer thickness, %
$\bnabla$ the horizontal gradient, %
$\nu$ the kinetic viscosity, %
$f$ the Coriolis parameter %
and %
$c = \sqrt{gH_0}$, with $g$ the gravitational acceleration.
%
In the following, the Coriolis parameter is assumed to be spatially
homogeneous.



\subsection{Eigenmodes of the linearized equations}

\label{subsection_eigen}

Neglecting the non-linear and the dissipative terms, the governing
equations can be rewritten as
\begin{equation}
\p_t q = 0, \mbox{\hspace{1cm}} 
\p_t d = a \mbox{\hspace{6mm} and \hspace{6mm}} 
\p_t a = - c^2\varkappa^2 d, 
\end{equation}
where $q = \zeta - f\eta$ is the linear Charney potential vorticity, %
$d = \bnabla \cdot \uu$ the horizontal divergence, %
$a = -c^2\bnabla^2 \eta + f\zeta$ an ageostrophic variable and %
$\varkappa^2 = {k_d}^2 - \bnabla^2$ the Helmholtz operator, %
with $k_d = f/c$ the deformation wave number. %
Here, $\eta = h-1$ is the normalized surface displacement and %
$\zeta = \eez \cdot (\bnabla \wedge \uu)$ is the vertical component of
the vorticity.
%
The dispersion relation is $\omega^2 = {\omega_l}^2$, where
\begin{equation}
 \omega_l(\kk) \equiv c  \sqrt{ {k_d}^2 + |\kk|^2 } = \sqrt{ f^2 + c^2|\kk|^2 },
\end{equation}
which implies that in the non-rotating case the waves are
non-dispersive.
%
The eigenfunctions of the linear operator are the prograde and
retrograde linear waves with positive and negative frequencies,
respectivelly.
%
Denoting the prograde quantities by the index $+$, the retrograde
quantities by the index $-$ and the temporal and spatial Fourier
transform by a tilde, we have by definition %
$a = a_+ + a_-$, $d = d_+ + d_-$, %
$\widetilde{\p_t a_\pm} = \mp i \omega_l \tilde a_\pm$ and %
$\widetilde{\p_t d_\pm} = \mp i \omega_l \tilde d_\pm$.
%
The linearized governing equations for the waves can be rewritten as %
$i \tilde a_\pm = \pm \omega_l \tilde d_\pm$, which gives %
$ \tilde a_\pm = \tilde a /2 \pm \omega_l \tilde d /(2i) .$




\subsection{Conserved quantities}


The non-dissipative one-layer shallow-water equations conserve a local
quantity along the trajectories of the fluid particles, the Ertel
potential vorticity $\ErtelPV = \zeta_a/h$, where $\zeta_a = \zeta +
f$ is the absolute vorticity.
%
Note that since the lagrangian derivative of the Ertel potential
vorticity is $\D_t \ErtelPV = 0$, the space-averaged Ertel potential
vorticity is not conserved $\p_t \meanx{\ErtelPV} = \meanx{d \ErtelPV}
\neq 0 $, where the brackets $\meanx{}$ denote the space average.
%
In the case of two-dimensional turbulence, it can be shown that the
energy can not cascade towards small scales due to the 
constraints that the enstrophy $\meanx{ \zeta^2/2 }$ is conserved and
that the spectra of energy and enstrophy are proportional
\cite[]{Kraichnan1967}.
%
This result was further extended by \cite{Charney1971} to the case of
quasi-geostrophic turbulence using the analogy between the
enstrophy and the Charney linear potential enstrophy $\meanx{ q^2/2
}$, where $q$ is the linear potential vorticity equal in the shallow
water case to $q = \zeta - f \eta$.
%
Note, however, that the analogy does not hold for the nonlinear Ertel
potential vorticity $\ErtelPV$ and that the local conservation of
$\ErtelPV$ is a much weaker constraint for the dynamics of the flows.


The non-dissipative one-layer shallow-water equations also conserve
some space-averaged quantities.
%
For example, since
\begin{equation}
(\D_t+d) \zeta_a = 0 \Rightarrow \p_t \meanx{\zeta_a} = 0 ,
\end{equation}
the space-averaged absolute vorticity $\meanx{\zeta_a }$ is conserved.
Since the governing equation for the thickness $h$ and the mass flux
$\JJ = h \uu$ are
\begin{eqnarray}
(\D_t+d) h 
&=& 0, \label{eq_h_2}\\
(\D_t+d) \JJ 
&=& - \bnabla E_P - f \eez \wedge \JJ, \label{eq_JJ}
\end{eqnarray}
where $E_P = c^2 h^2/2$ is the total potential energy, the
space-averaged thickness $\meanx{ h }$ is conserved and the kinetic
momentum $\meanx{ \JJ }$ is conserved in the absolute not-rotating
reference frame.
%
Note that in contrast to the case of the Navier-Stokes equations, 
the dissipative shallow-water equations do not conserve $\meanx{ \JJ }$,
even with a Newtonian viscous operator.

The non-dissipative one-layer shallow-water equations also conserve
the space-averaged total energy, the local energy being the sum of the
local potential energy (PE), $E_P = c^2 h^2/2$, and the local kinetic
energy (KE), $E_K = \JJ\cdot\uu/2$.  Using (\ref{eq_uu}),
(\ref{eq_h_2}) and (\ref{eq_JJ}), it is straightforward to show that
\begin{eqnarray}
(\D_t + d) E_K
&=&  
- \uu \cdot \bnabla E_P, \\
(\D_t + d) E_P
&=&  
- E_P \bnabla \cdot \uu.
\end{eqnarray}
%
The space-averaged conversion from potential energy to kinetic energy
is equal to $C = -\meanx{ \uu \cdot \bnabla E_P } = \meanx{ E_P
\bnabla \cdot \uu }$.
%
The total potential energy can be split in three parts
\begin{equation}
E_P = c^2/2 + c^2\eta+ c^2 \eta^2/2.
\end{equation}
The first term corresponds the potential energy of the state with null
surface displacement.  The second term is a contribution to the
potential energy which is zero in average.  We call the third term
$E_A = c^2\eta^2/2$ the available potential energy (APE).  Its
space-average is indeed equal to the space-averaged APE.
