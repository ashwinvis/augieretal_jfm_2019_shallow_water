


\subsection{Analysis of the wave energy cascade}

\subsubsection{Formulation of the spectral energy budget for
non-quadratic energy}

In order to analyze the flux of energy, we derive the spectral energy
budget, i.e.\ the governing equations for spectral KE and APE
functions.
%
This is not as straightforward as in incompressible turbulence since
the expression of the kinetic energy $E_K = \JJ\cdot\uu/2$ is not
quadratic.
%
In order to define the spectral KE function, we use the relation
\begin{equation}
\meanx{ E_K } = \sum_\kk \scalarprod{\uu}{\JJ}/2,
\end{equation}
where
\begin{equation}
\scalarprod{\baa}{\bb} \equiv \Re\{ \hat \baa(\kk)^* \cdot \hat \bb(\kk) \},
\end{equation}
where $\Re$ denotes the real part, $\kk$ is the wave number and the
hat denotes the Fourier transform.
%
The spectral KE function can therefore be defined as
\begin{equation}
E_K(\kk) \equiv \scalarprod{\uu}{\JJ}/2, 
\end{equation}
such as $\meanx{ E_K } = \sum_\kk E_K(\kk)$.
%
Similarly, the PE can be written as the sum over all wave numbers of
the spectral PE function $E_P(\kk) = c^2 |\hat h|^2/2 = c^2
\scalarprod{h}{h}/2$, where, by definition,
\begin{equation}
\scalarprod{a}{b} \equiv \Re\{ \hat a(\kk)^* \hat b(\kk) \}.
\end{equation}


The equation for the spectral KE and PE functions can be written as
\begin{eqnarray}
\p_t E_K(\kk) &=& T_K(\kk) + C_K(\kk),\\
\p_t E_P(\kk) &=& T_P(\kk) - C_P(\kk),
\end{eqnarray}
where
\begin{equation}
C_K(\kk) 
= -\scalarprod{\uu}{\bnabla E_P}/2 - c^2 \scalarprod{\JJ}{\bnabla h}/2,
\end{equation}
\begin{equation}
C_P(\kk) = c^2 \scalarprod{h}{hd} /2,
\end{equation}
\begin{equation}
T_P(\kk) = -c^2 \scalarprod{h}{\uu \cdot \bnabla h + hd/2}
\end{equation}
and
\begin{equation}
T_K(\kk) 
= -\scalarprod{\uu}{\uu \cdot \bnabla \JJ}/2
-\scalarprod{\JJ}{\uu \cdot \bnabla \uu}/2
-\scalarprod{\uu}{d\JJ}/2.
\end{equation}
The mean conversion from potential energy to kinetic energy can be
computed from the two conversion spectral functions as $C = \sum_\kk
C_K(\kk) = \sum_\kk C_P(\kk)$.






\subsubsection{Exact Kolmogorov law for irrotational flows}



As in incompressible homogeneous isotropic turbulence, an exact
Kolmogorov law for irrotational flows can be derived.
%
It is convenient to start from the following form of the governing
equations
\begin{eqnarray}
\p_t \uu &=& -\bnabla \cdot (|\uu|^2/2) - c^2 \bnabla h  
- \zeta_a \eez \wedge \uu, \\
\p_t \JJ &=& -\p_j (u_j \JJ) - \bnabla E_p,\\
\p_t h   &=& -\bnabla \cdot \JJ.
\end{eqnarray}
Assuming that the flow is irrotational and that there is no global
rotation ($\zeta_a=0$), it can be shown that
\begin{equation}
\p_t ( J_i' u_i ) = - \p_i( J_i'|\uu|^2/2) - \p_i'(u_j J_j' u_i')
-\p_i (J_i' c^2 h)  - \p_i' (u_i E_p')
\end{equation}
and
\begin{equation}
\p_t ( h' h ) = - \p_i( h'J_i ) - \p_i'( hJ_i' ),
\end{equation}
which gives
\begin{eqnarray}
\p_t ( \JJ'\cdot\uu + \JJ\cdot\uu' + 2c^2 h'h )
&=& - \p_i( J_i'|\uu|^2/2) - \p_i'( J_i|\uu'|^2/2) %\nonumber \\&& 
- \p_i'(u_j J_j' u_i') - \p_i(u_j' J_j u_i) \nonumber\\
&& -c^2\p_i (J_i'h) - c^2\p_i' (J_ih') %\nonumber\\&& 
- \p_i' (u_i E_p') - \p_i (u_i' E_p) \nonumber\\
&& -2c^2\p_i' (J_i'h) - 2c^2\p_i (J_ih').
\end{eqnarray}
We then assume homogeneity and take the ensemble average denoted by
$\meane{}$.  The separation vector is noted $\rr = \xx' - \xx$.  For
every function $g(\xx, \xx')$, we have $\p_i' \meane{g} =
-\p_i\meane{g} = \bnabla_{\rr} \meane{g}|_i$. We obtain
\begin{eqnarray}
\p_t \meane{ \JJ'\cdot\uu + \JJ\cdot\uu' + 2c^2 h'h }
&=& + \bnabla_{\rr}\cdot\meane{ \JJ'|\uu|^2/2 - \JJ|\uu'|^2/2 } \nonumber\\
&& - \bnabla_{\rr}\cdot\meane{ u_j J_j' \uu' - u_j' J_j \uu } \nonumber\\
&& - \bnabla_{\rr}\cdot\meane{ \uu E_p'- \uu' E_p } \nonumber\\
&& -c^2\bnabla_{\rr}\cdot\meane{ \JJ'h - \JJ h' }
\end{eqnarray}
%
We then introduce the structure functions and the operator $\delta$
returning the increment of a variable between two points separated by $\rr$, 
for example $\delta h (\xx, \rr) = h(\xx+\rr)- h(\xx)$.
%
Using again homogeneity, we get
\begin{equation}
\meane{ (\delta h)^2 \delta\uu } = 
-\meane{ h'^2 \uu } + \meane{ h^2 \uu' }
+2\meane{ hh' \uu } - 2\meane{ hh' \uu' }
\end{equation}
and 
\begin{equation}
\meane{ |\delta \uu|^2 \delta\JJ } = 
\meane{  |\uu|^2 \JJ' } - \meane{ |\uu'|^2\JJ }
+\meane{ u_ju_j' \JJ' } - \meane{  u_ju_j' \JJ },
\end{equation}
which gives
\begin{equation}
2\p_t \meane{ \JJ'\cdot\uu + \JJ\cdot\uu' + 2c^2 h'h }
= \bnabla_{\rr} \cdot ( \meane{ |\delta \uu|^2 \delta\JJ } 
+ c^2\meane{ (\delta h)^2 \delta\uu } ).
\end{equation}
Using isotropy and integrating yield an exact Kolmogorov law for
one-layer shallow water irrotational turbulence
\begin{equation}
\meane{ |\delta \uu|^2 \delta J_L } 
+ c^2\meane{ (\delta h)^2 \delta u_L } = -4 \varepsilon r, \label{eq_Kolmo}
\end{equation}
where $\eps$ is the energy dissipation rate and $J_L \equiv
\JJ\cdot\rr / |\rr|$ and $u_L \equiv \uu\cdot\rr / |\rr|$ are
longitudinal increments.







\subsection{Simple model for structure functions determined by a
random set of shocks}
\label{subsection_shock_model}


\cite{BouchaudMezardParisi1995} and \cite{Kuznetsov2004} showed that
the presence of discontinuities can strongly influence the statistics
and in particular lead to very strong intermittency.
%
We now present an extremely simple model based on the assumption that
shocks totally dominate the flow.  The structure functions are
calculated by averaging increments over the total surface of the
domain $\mA$.  Considering that the increments are produced only by
discontinuity lines, we get
\begin{eqnarray}
\meane{|\delta h|^p} 
&=& \frac{1}{\mA} \int_{\mA}  d^2\xx |\delta h(\xx)|^p \nonumber\\
&\simeq& \frac{r}{\mA} \int_\shocks ds |\sin\theta| |\Delta h (s)|^p,
\end{eqnarray}
where $\theta$ is the angle between the shock line and the separation
vector $\rr$ %
and $\Delta h (s)$ is the step in $h$ at a shock.
%
Assuming isotropy, the integral can be split and we get
\begin{equation}
\meane{|\delta h|^p} 
\simeq 
r \frac{L_s}{\mA} \meant{|\sin\theta|} \means{|\Delta h (s)|^p}
=
r \frac{L_s}{\mA} \frac{2}{\pi} \means{|\Delta h (s)|^p},
\end{equation}
where $\meant{}$ denotes the mean over $\theta$, %
$L_s$ is the total length of the shocks in the domain %
and $\means{}$ denotes the mean over all shocks.

For velocity increments, we can also use a characteristic of the
velocity singularities.  The step in velocity is only in the component
perpendicular to the shocks, which implies that the longitudinal and
transverse increments are related by $\delta u_L = \delta u
\sin\theta$, $\delta u_T = \delta u \cos\theta$.
%
Applying the same averaging method, we obtain
\begin{equation}
\meane{|\delta u_L|^p} 
\simeq 
r \frac{L_s}{\mA} 
\meant{|\sin\theta|^{1+p}}
\means{|\Delta u (s)|^p}
\end{equation}
and
\begin{equation}
\meane{|\delta u_T|^p} 
\simeq 
r \frac{L_s}{\mA} 
\meant{|\sin\theta||\cos\theta|^{p}}
\means{|\Delta u (s)|^p},
\end{equation}
where $\Delta u (s)$ is the velocity step at a shock.
%
These assumptions completely determine the ratio between the
longitudinal and transverse structure functions, which can be
analytically computed as
\begin{eqnarray}
R_p(r) \equiv \frac{\meane{|\delta u_L|^p}}{\meane{|\delta u_T|^p}} = 
\frac{\meant{|\sin\theta|^{1+p}}}{\meant{|\sin\theta||\cos\theta|^p}},
\end{eqnarray}
giving the numerical values $R_2 = 2$, $R_3 = 6\pi/8$ and $R_4 = 8/3$.


Interestingly, for an isotropic and purely divergent flow ($\zeta =
0$), the longitudinal and transverse second order structure functions
are exactly related by $\meane{(\delta u_L)^2} = \p_r( r\meane{(\delta
u_T)^2} )$ \cite[]{Lindborg2007jas}.
%
If we further assume than both second order structure functions follow
the same scaling law $\meane{ (\delta u_L)^2} \propto \meane{ (\delta
u_T)^2} \propto r^\alpha$, then we have $\meane{ (\delta u_L)^2} =
(1+\alpha) \meane{ (\delta u_T)^2} $, which gives $ R_2 = 2
\Leftrightarrow \alpha = 1$.


The shock model also provides predictions of the flatness factors of
the velocity increments
\begin{eqnarray}
F_L &=&  \frac{\meane{|\delta u_L|^4}}{\meane{|\delta u_L|^2}^2} 
= r^{-1} \frac{\means{|\Delta u|^4}}{\means{|\Delta u|^2}^2} 
\frac{\meant{|\sin\theta|^5}}{\meant{|\sin\theta|^3}^2},\\
F_T &=&  \frac{\meane{|\delta u_T|^4}}{\meane{|\delta u_T|^2}^2} 
= r^{-1} \frac{\means{|\Delta u|^4}}{\means{|\Delta u|^2}^2} 
\frac{\meant{|\sin\theta||\cos\theta|^4}}{\meant{|\sin\theta||\cos\theta|^2}^2},
\end{eqnarray}
and on the ratio of the two flatness factors, which is simply equal to
$F_T/F_L = 1.5$ according to the model.






