


In order to study the effects of system rotation, we have carried
out four supplementary simulations for $c = 20$, $n = 1920$ and
constant and non-zero Coriolis parameter $f$.  The values of $f$ have
been chosen such that to yield relatively small Rossby numbers
\begin{equation}
Ro_f = \frac{\eps^{1/3} {k_f}^{2/3}}{f} \lesssim 0.1,
\end{equation}
and to span a range of Burger number,
\begin{equation}
Bu = \left(\frac{k_f}{k_d}\right)^2 = \left(\frac{Ro_f}{F_f}\right)^2,
\end{equation}
going from 0.5 to 4.  For these moderate values of global rotation, a
statistically stationary regime is also reached.  Table~\ref{tab_rot}
displays the parameters for these simulations.
\begin{table}
\begin{center}
\begin{tabular}{cc@{\hskip 8mm}c@{\hskip 8mm}ccc@{\hskip 8mm}cccc@{\hskip 8mm}cc}
$n$ & $c$ & $\nu_8$ & $f$ & $Ro$ & $Bu$ & $\eps$ & $\displaystyle\frac{\kmax}{\kdiss}$ & $\displaystyle\frac{\kdiss}{k_f}$ & $F_f$ & $\min h$ & $\displaystyle\frac{\max |\uu|}{c}$ \\[3mm]
1920 &   20 & 9.6e-13 &  0   &  $\infty$ & $\infty$ & 0.99 & 2.46 &  58 &  0.055 & 0.59 & 0.56 \\
1920 &   20 & 9.6e-13 &  7.5 &   0.11 &      4 & 0.96 & 2.47 &  58 &  0.054 & 0.67 & 0.52 \\
1920 &   20 & 9.6e-13 & 10.7 &  0.076 &      2 & 0.93 & 2.47 &  58 &  0.054 & 0.70 & 0.62 \\
1920 &   20 & 9.6e-13 & 15.1 &  0.052 &      1 & 0.85 & 2.48 &  57 &  0.052 & 0.70 & 0.65 \\
1920 &   20 & 9.6e-13 & 21.3 &  0.037 &    0.5 & 0.84 & 2.48 &  57 &  0.052 & 0.66 & 0.81 \\
\end{tabular}
\caption{Overview of parameters for the simulations used to study the effect of rotation. 
}
\label{tab_rot}
\end{center}
\end{table}



\begin{figure}
% \centerline{
% \includegraphics[width=\halfwidth]{../Figs/fig_Emean_f}
% \includegraphics[width=\halfwidth]{../Figs/fig_spectra_f_Nx=1920}
% }
\caption{Effect of the rotation: (\textit{a}) mean energy versus
$k_d/k_f$ and (\textit{b}) energy spectra for $c = 20$ and $n =
1920$.}
\label{fig_effectBu}
\end{figure}

Figure~\ref{fig_effectBu}(\textit{a}) shows the mean energy as a
function of the Burger number.  We see that for $Bu = 2$ and 4, the
mean energy is very close to the value obtained for the non-rotating
case.
%
However, when $k_f<k_d$, i.e. for $Bu<1$, the mean energy increases.
%
Figure~\ref{fig_effectBu}(\textit{b}) presents the spectra for the
same simulations.  The spectra for $Bu = 2$ and 4 are very close to
the spectra for $f= 0$, which confirms that a weak rotation does not
deeply modify the non-rotating results.
%
We have also verified that for the moderate rotation rates
corresponding to $Bu>1$, the other results presented in
section~\ref{section_wave_cascade_f=0} on the non-rotating wave
cascade are only weakly modified.  In particular, the results for $Bu
= \infty$ and $Bu = 4$ are very similar.
